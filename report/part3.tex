% !TEX root=main.tex

\section{Part 3: Control system design}

\subsection{3a)}

We want to design a limited PD controller with the transfer function

\begin{equation}
    H_{pd}(s) = K_{pd}\frac{1+T_ds}{1+T_fs} \label{eq:H_pd}
\end{equation}

to control the heading $\psi$ of the ship, by setting the rudder angle $\delta$. We want the cross frequency and phase margin of the open loop system to be $0.10\si{\radian\per\second}$ and $50$ degrees respectively. We also want the time constant $T_d$ to equal the time constant of the system. In order to choose $K_{pd}$, $T_d$ and $T_f$ we first find the transfer function of the open loop system

\begin{equation}
    H_{sys}(s) = H_{pd}(s) \cdot H_{ship}(s) = KK_{pd}\frac{1+T_ds}{s(Ts+1)(T_fs+1)} \label{eq:H_sys}
\end{equation}

As we want $T_d$ to equal the time constant of the model we set $T_d = T = 86.5246 \si{\second}$. Next we use that the phase margin of a system is equal to $\angle H(j\omega_c) + 180 \si{\degree}$.

\begin{subequations}
    \begin{align}
        \angle H_{sys}(j\omega_c) + 180 \si{\degree} &= 50 \si{\degree} \\
        \angle \frac{KK_{pd}}{(j\omega_c)^2T_f+j\omega_c} &= -130 \si{\degree}\\
        \angle KK_{pd} - \angle ((j\omega_c)^2T_f+j\omega_c) &= -130 \si{\degree} \\
        \angle ((j\omega_c)^2T_f+j\omega_c) &= -130\si{\degree} \\
        \tan^{-1}\frac{\omega_c}{\omega_c^2T_f} &= 50 \si{\degree} \\
        T_f &= \frac{1}{ \omega_c \tan (50 \si{\degree}) } \\
        T_f &= 8.3910 \si{\second} \label{eq:T_f}
    \end{align}
\end{subequations}

Last we find $K_{pd}$ by using the definition of the cross frequency: $|H(j\omega_c)| = 1$. This yields

\begin{subequations}
    \begin{align}
        |H_{sys}(j\omega_c) &= 1 \\
        |H_{sys}^2(j\omega_c) &= 1^2 \\
        \left | \frac{KK_{pd}}{s^2T_f+s} \right |^2 &= 1 \\
        \frac{(KK_{pd})^2}{(\omega_c^2 T_f)^2 + \omega_c^2} &= 1 \\
        K_{pd} &= \frac{\sqrt{(\omega_c^2 T_f)^2} + \omega_c^2}{K} \label{eq:K_pd(T_f)}
    \end{align}
\end{subequations}

Inserting \cref{eq:T_f} into \cref{eq:K_pd(T_f)} yields a $K_{pd} = 0.7494$. The controller was implemented in Simulink as seen in \cref{fig:pd}.

\begin{figure}
    \centering
    \includegraphics[width=\textwidth]{images/oppg3/a_pd-loop.pdf}
    \caption{Simulink implementation of the PD-controller}
    \label{fig:pd}
\end{figure}

\subsection{3b)}

As we can see in \cref{fig:step_no_dist} the controller does a good job when there are no disturbances. The ship reaches the reference $\psi_r$ of $30\si{\degree}$ within 2 minutes, and the rudder angle is relatively constant over time.

\begin{figure}
    \centering
    \includegraphics[width=\textwidth]{images/oppg3/stepresp_no_disturbance.eps}
    \caption{Step response of the controller. No disturances.}
    \label{fig:step_no_dist}
\end{figure}

\subsection{3c)}

In \cref{fig:step_current_dist} we see the response of the controller with current disturbance. as we can see the controller is unable to counteract the constant disturbance of the current, which leads to a large constant deviation from the setpoint.

\begin{figure}
    \centering
    \includegraphics[width=\textwidth]{images/oppg3/stepresp_current_disturbance.eps}
    \caption{Step response of the controller. With current disturbance.}
    \label{fig:step_current_dist}
\end{figure}

\subsection{3d)}

When we apply wave current (and turn off current disturbance), we see that the controller does it best to remove the noise but is unable to do much. In its attempt to remove the noise it applies a lot of rudder input, which if the boat was real would stress the physical system.

\begin{figure}
    \centering
    \includegraphics[width=\textwidth]{images/oppg3/stepresp_wave_disturbance.eps}
    \caption{Step response of the controller. With wave disturbance}
    \label{fig:step_wave_dist}
\end{figure}
