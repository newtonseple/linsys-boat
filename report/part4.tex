% !TEX root=main.tex

\section{Part 4: Observability}

\subsection{4a)} \label{subsec:4a}

The model in \cref{eq:completeModel} can be written on the form $\mathbf{\dot x} = \mathbf{A} \mathbf{x} + \mathbf{B}u + \mathbf{E}\mathbf{w}$, $ y = \mathbf{C}\mathbf{x} + v$, with $\mathbf{x} = [ \xi_w\ \psi_w\ \psi\ r\ b]^ T $ , $ u = \delta $ and $ \mathbf{w} = [w_w\ w_b]^T$. Then the matrices $\mathbf{A}$, $\mathbf{B}$, $\mathbf{C}$ and $\mathbf{E}$ are

\begin{subequations}
    \begin{align}
        \mathbf{A} &= \begin{bmatrix}
        0 & 1 & 0 & 0 & 0 \\
        -\omega_0^2 & -2\lambda\omega_0 & 0 & 0 & 0 \\
        0 & 0 & 0 & 1 & 0 \\
        0 & 0 & 0 & -1/T & -K/T \\
        0 & 0 & 0 & 0 & 0
        \end{bmatrix} \\
        \mathbf{B} &= \begin{bmatrix}
        0 \\
        0 \\
        0 \\
        K/T \\
        0 \\
        \end{bmatrix} \\
        \mathbf{E} &= \begin{bmatrix}
        0 & 0 \\
        K_w & 0 \\
        0 & 0 \\
        0 & 0 \\
        0 & 1 \\
        \end{bmatrix} \\
        \mathbf{C} &= \begin{bmatrix}
        0 & 1 & 1 & 0 & 0
        \end{bmatrix}
    \end{align}
\end{subequations}

\subsection{4b)} \label{subsec:4b}

 We first checked observability without any disturbance. To do this we removed the disturbances from the system. the resulting $\mathbf{A}$ and $\mathbf{C}$ matrices were

\begin{subequations}
    \begin{align}
        \mathbf{A} &= \begin{bmatrix}
        0 & 1 \\
        0 & -1/T
        \end{bmatrix} \\
        \mathbf{C} &= \begin{bmatrix}
        1 & 0
        \end{bmatrix}
    \end{align}
\end{subequations}

To find out whether the system is observable we checked if \texttt{rank(obsv(A, C)) == size(A,1)}. This code calculates the rank of the observability matrix $\mathcal{O}$, and compares it to the dimension of the matrix $\mathbf{A}$. Using this code we found that the system is observable without any disturbance.

\subsection{4c)}

Next we checked observability with the current disturbance. The $\mathbf{A}$ and $\mathbf{C}$ matrices used were

\begin{subequations}
    \begin{align}
        \mathbf{A} &= \begin{bmatrix}
        0 & 1 & 0 \\
        0 & -1/T & -K/T \\
        0 & 0 & 0
        \end{bmatrix} \\
        \mathbf{C} &= \begin{bmatrix}
        1 & 0 & 0
        \end{bmatrix}
    \end{align}
\end{subequations}

Using the code in \cref{subsec:4b} we found that the system is observable with current disturbance.

\subsection{4d)}

Next we checked observability with the wave disturbance. The $\mathbf{A}$ and $\mathbf{C}$ matrices used were

\begin{subequations}
    \begin{align}
        \mathbf{A} &= \begin{bmatrix}
        0 & 1 & 0 & 0 \\
        -\omega^2 & -2\lambda\omega_0 & 0 & 0 \\
        0 & 0 & 0 & 1 \\
        0 & 0 & 0 & - 1/T
        \end{bmatrix} \\
        \mathbf{C} &= \begin{bmatrix}
        0 & 1 & 1 & 0
        \end{bmatrix}
    \end{align}
\end{subequations}

Using the code in \cref{subsec:4b} we found that the system is observable with wave disturbance.

\subsection{4e)}

Next we checked observability with both current and wave disturbance. The $\mathbf{A}$ and $\mathbf{C}$ matrices used were

\begin{subequations}
    \begin{align}
        \mathbf{A} &= \begin{bmatrix}
        0 & 1 & 0 & 0 & 0 \\
        -\omega_0^2 & -2\lambda\omega_0 & 0 & 0 & 0 \\
        0 & 0 & 0 & 1 & 0 \\
        0 & 0 & 0 & -1/T & -K/T \\
        0 & 0 & 0 & 0 & 0
        \end{bmatrix} \\
        \mathbf{C} &= \begin{bmatrix}
        0 & 1 & 1 & 0 & 0
        \end{bmatrix}
    \end{align}
\end{subequations}

Using the code in \cref{subsec:4b} we found that the system is observable with both wave and current disturbance.
